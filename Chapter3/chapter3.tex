%!TEX root = ../thesis.tex
%*******************************************************************************
%****************************** Third Chapter **********************************
%*******************************************************************************
\chapter{Results}
\section{Connectomics}
\subsection{The Central Complex Adult vs Larva}
\subsection*{Protocerebral Bridge}
In the adult \textit{Drosophila}, the PB comprises two sets of bilaterally symmetric compartments, sometimes referred to as glomeruli 1–9 ~\cite{hulse2021connectome}, positioned at the most posterior-dorsal location possible in the brain.

These compartments are arranged in a continuous manner medio-laterally, contacting at the midline.
In the adult, about 600 neurons innervate the PB, organised into hundreds of types (194; \citep{wolff2015neuroarchitecture}) that are split into two main general groups: the columnar neurons (from lineages DM1, DM2, DM3, DM4 and DM6) whose dendrites innervate one or more of the 9 + 9 compartments of the PB~\citep{wolff2015neuroarchitecture}; and the horizontal neurons (also known as horizontal fibers) derived from a single lineage (PBp1; \citep{andrade2019developmentally}) whose axons innervate many or all PB compartments.

In the adult, the PB receives visual input via relay neurons (POL neurons) conveying information on polarized light, in a highly structured pattern across its compartments that binarizes the continuum of angles of polarized light (\citep{heinze2009transformation}). Then the PB relays this information to the EB compartments.
% (Table \ref{inputsoutputs});

In addition to visual input, the adult PB also integrates olfactory inputs \citep{hulse2021connectome}, suggesting that spatial navigation is not unimodal but integrative across multiple sensory modalities.

In searching for the larval PB, we expected two sets of neurons: columnar and horizontal. In larva, four central complex lineages contribute columnar neurons, a subset of which position their dendrites at a posterior-dorsal location. We could not find a central complex lineage that would contribute horizontal fibers at a posterior-dorsal location necessary to intersect and synapse onto the dendrites of the PB columnar neurons, but we found a larval lineage (DALv1) whose axons are bilateral and project to the appropriate area, and is developmentally related to another central complex lineage (DALv23). This suggests that neurons from non-central complex lineages may be recruited temporarily during the larval period, in a pattern reported so far for the mushroom body (see Discussion; \citep{truman2023metamorphosis}). 

% QUESTION TODO: do columnar neurons of the PB only contribute dendrites to the PB? 
%ANSWER: Tanya wolff's papers say yes. Hulse et al fig 29 supplement 1 & Fig 33:  PFNm_a, PFNm_b 
%unclear where the pfn_m connection is => answer is yes 


Among neurons of the DALv1 lineagel, 4 left-right pairs (named HF-PB for "Horizontal Fiber PB") project their axons bilaterally and across the dendrites of the columnar neurons.
3 of the 4 pairs present an unusual axon configuration: first, they project contralaterally to drop their first output synapses, with the axon then crossing the midline a second time to return back to the same ipsilaterally corresponding location to again drop presynaptic sites (\ref{FigureX}).
This peculiar axon configuration is unique among all neurons of the entire brain of the larva~(\citep{winding2023}) and suggestive of potentially a delay line for comparing left-right sensory inputs.
The 4th pair first drops presynaptic sites ipsilaterally and then its axon crosses the midline until reaching the corresponding contralateral location to synapse again~(\ref{FigureX}).
% Relative to the root of their dendritic tree, which is ipsilateral, the first set of output synapses are found contralaterally. Then the axon crosses the midline a second time, returning to the ipsilateral hemisphere to establish more output synapses in a spatially symmetric way to the contralateral set of output synapses.
The presynaptic outputs of DALv1 neurons are symmetric, in that they contact the same homologous pairs of left-right neurons which are predominantly neurons of the columnar system ~(\ref{FigureX}).
The axons of these 4 pairs of HF-PB neurons are tiled dorso-ventrally, falling into two bilaterally symmetric groups which we interpret as defining 2 + 2 bilaterally arranged PB compartments, each innervated by 2 pairs of axons. % figure PB compartments

The dendrites of these 4 pairs of DALv1 neurons (HF-PB) are ipsilateral and dorsal, receiving polysynaptic inputs from vision and olfaction, like in the adult PB~(\citep{hulse2021connectome}). In the larva, we found that these multi-sensory inputs to the horizontal fibers of the PB are mediated by Convergence Neurons (CN-53 and CN-54, among others; \citealp{eschbach2021}) that, as their name indicates, integrate inputs from both Mushroom Body Output Neurons (MBONs) and from the Lateral Horn (LH) such as olfactory and visual PNs \citep{EsbachFushiki2021}). This circuit architecture indicates that sensory inputs arriving to the larval PB will have been modulated or gated by previously established associative memories, with implications for spatial navigation.
% check for more CNs or other neurons converging onto PB DALv1 dendrites
%ANSWER: CN35, CN45, CN9 low syn count CN26, CN15
%as well as from other sensory modalities (Table \ref{sensoryinputs})


%TODO: describe dendritic and axonic contributions in specific (especially from AL, OL and MB) 

%2 systems: pattern of input of dendrites of the DALv1 s.
% MB2ON-175 & 48 & 208 
% given the PB1-4 compartments, which ones are connected to the MBONs, and the CNs. 175 targets one compartment, 208 targets another.  The pattern is that CNs connect to one compartment of the PB while simultaneously connecting to another CN that connects to the opposite compartment - there could be a gating mechanism. 
%MB2ON 202 - feedback from the NO to the PB dalv1 


In the larva, the columnar system consists of neurons from 4 central complex lineages (DPMpm1, DPMpm2, DPMm1 and CM4) that also generate the columnar neurons of the adult (DM1, DM2, DM3 and DM4, correspondingly).
Larval columnar neurons present small, narrow dendrites circumscribed within the 2 + 2 compartments defined by the axons of the horizontal fibers (DALv1 neurons), with whom they synapse.
Among the columnar neurons, a subset project their axons directly to the Noduli (NO; \ref{FigureX}), and another subset project directly to the larval Ellipsoid Body (EB; \ref{FigureX}).
We did not find in the larva columnar neurons whose axons would project to more than one Central Complex neuropil, despite such types being common in the adult~\citep{wolff2015neuroarchitecture; wolff2018; hulse2021connectome}.
Beyond the canonical columnar neurons projecting to other Central Complex neuropils, we found some whose axons descend to the SEZ or nerve cord~(\ref{FigureX}).



% TODO Discussion point: this is different than in the adult. 

% QUESTION: in the adult, are there columnar neurons that descend to the nerve cord? We have them too in larvae Should mention I them. %ANSWER: see figure 63A Hulse et al. supplement 1 PFL -> MDNs (I have a feeling mdn is smth else in adult)

%In the \textit{Drosophila} larva Electron Microscopy (EM) volume, we found a putative PB situated at the most medial-posterior side of the brain, that receives high levels of visual input and is made of neurons belonging to lineages DM2 (DPMpm1 in the larva) and 8 columnar neurons(4 pairs) belonging to the lineage DALv1. 
% also - a dopaminergic pathway formed by large field CIVP neurons that relay IDFP signals to the entire PB


\subsection*{Ellipsoid Body}
The adult Ellipsoid Body(EB) is a ring-shaped structure situated between the Fan-Shaped Body(FB) and the Mushroom Body horizontal lobes, facing anterodorsally. Its circuit is made up two types of neurons: ring-neurons (derived mainly from the EBAa1/DALv2 and LALv1/BAmv1/2 lineages) that spread their axons across the length of the EB, and reciprocally connected wedge neurons(derived from the DALcl12 lineage) that divide the EB into 16 compartments (aka. wedges)~\citep{omoto2018neuronal}. 

Its underlying circuit follows the ring attractor architecture (Zhang, 1996) which, as predicted by its anatomy, is shown to yield neural activity in the form of a topological ring in \textit{Drosophila} adult(Seeling \& Jayaraman 2015) with all nodes being connected via inhibitory connections, complemented by local recurrent excitations that maintain activity at each node once they escape inhibition.%(this last sentence is Stanley Heinze).

The wedge neurons(EPG) form eight wedges around this ring, and project to both hemispheres of the PB, where they connect to two sets of columnar neurons that project back to the EB, forming recurrent loops. These are PEG and PEN neurons. The anatomical offset between EPG and PEN neurons is key to how the fly head direction system translates angular motion into an updated position of the activity bump in the ring attractor. 


The EB receives visual inhibitory GABAergic inputs, via two parallel pathways for distinct visual information:
1. Ring neurons that deconstruct the visual environment of the fly; 2. tangential neurons that take in information about body rotations and transnational velocity. The latter receive input in the LAL, output to NO.
Mechanosensory input also enters the CX via the second order projection neurons to the EB. These neurons code head direction; some proprioceptive input has also been observed 
~\citep{hulse2021connectome}. It receives strong inputs from PB, NO and the LAL, and outputs onto the PB.  


In the 1st instar larva, we found a group of 8 pairs of reciprocally connected neurons from lineage DALcl12 known to produce wedge-neurons in the adult, and categorised these together with one other pair of lineage Dalv23 (which produces ring neurons in adult) with the same connectivity pattern as wedge-neurons. Both their dendrites and axons are very small, and tiled medio-laterally, defining 8 compartments with one single neuron pair contributing to each. These are the intrinsic set of neurons, fully enclosed within the putative larval EB. 

Similarly, we found one pair of neurons of the BAmv1/2 lineage - known to contribute to ring neurons in adult flies - that receive visual input via PB neurons, and reciprocally interconnects with the previously mentioned wedge-neurons, and whose axons are fully contained within the space defined by the wedge neurons. We categorised these as larval "ring" neurons.

%We found that a putative structure for the EB in the larva with neurons coming from DALcl12, DALl1 and DALv2(3). This structure seems to receive input from both the LAL and NO(weak) and sends outputs to the PB.



\subsection*{Fan-Shaped Body}
% Constituent neurons and the horizontal and vertical compartments they define


The adult FB is a bilaterally symmetric neuropil anterior to the PB, with well-defined horizontal and vertical components: it has 6 horizontal layers stacked dorso-ventrally that are defined by distinct sets of horizontal neurons(FB tangential neurons); and 9 vertical columns stacked medio-laterally are defined by column-specific columnar neurons. Both horizontal and vertical neurons innervate the FB in a layer- and column-restricted manner~\citep{heinze2017unraveling}. As one of the biggest CX neuropils, a large variety of lineages contribute to the FB (see Table \ref{lineages}).
The FB does not receive input along only one clearly defined input pathway, but it is connected to many regions of the surrounding protocerebrum via tangential neurons. 

There are 2 types of FB tangential cells: (1)neurons that relay the presence of an attractive odor to the FB, originating in the MB or the LH (learnt or innate valences); (2) neurons that relay sleep drive to the FB, whose activity is mandatory for sleep initiation. 

The FB columnar neurons, or columnar input cells are known as PFN (PB-FB-NO) and they receive information both in the PB and in the Noduli output cells with dendritic fibers mainly in the FB; 


There are 5 types of PFNs, they form a p they all receive the same head direction input from the PB, which is integrated with different input signals received in the NO. The PFN outputs are located in distinct layers of the ventral/posterior FB, essentially mapping the noduli layers onto corresponding regions of the FB. PFN cells have a columnar projection pattern that is offset from the default projection scheme between the PB and the central body. This offset generates a head direction bump in the FB that is contralaterally shifted relative to the PB by one column, i.e., 45° of azimuthal space, thus separating right and left cells originating in corresponding PB columns by 90° in the FB.

The third class of FB cells are interneurons which input and output within the regions of the FB.There are 2 types: FB intrinsic neurons; FB mixed arborisation neurons with additional output branches outside the CX and sometimes input fibers in the PB.


% Typical synaptic inputs and outputs
A key feature of the the adult FB is strong innervation by Mushroom Body Output Neurons (MBONs)~\citep{MISSING}. % many, including hemibrain paper
In addition, the axons of dopaminergic neurons driven by visual inputs innervate the FB~\citep{lin2013comprehensive}.


% In larva, these are CNs or LHONs:
%Among the many other inputs to the FB, to remark inputs from the lateral horn (LH)~\citep{hulse2021connectome}, a region known to compute valences from multimodal inputs~\citep{StrutzSachse2014odorquality}. Within the CX, the FB forms bidirectional connections with the PB, the Noduli (NO), and the Lateral Accessory Lobe (LAL). 

% In larva, now describe FB.b (mostly horizontal fibers like the u-shaped horseshoe neurons) and the FB.d.

In the larva, we found a number of putative FB horizontal/tangential cells
originating in lineages known to contribute neurons to the adult FB. Characteristically, most present a bilateral axon closely wrapping around the midline, and an ipsilateral dendrite positioned within the superior dorsal protocerebrum (dorsal anterior neuropil) where they integrate numerous inputs from MBON axons. Among the various neurons with dendrites within this very medial neuropil, we find neurons from lineages known to contribute to the adult FB and whose axons project to the putative larval NO, EB, PB and LAL.

%We found that a putative FB is also present at the larva, with neurons from the lineages DPMpm1, CM13, DPMpl12 and DPMpm2, and which receives strong synaptic input from MBONs as well as strong reciprocal connectivity with the LAL, the putative NO, and the putative PB. 

\subsection*{Noduli}
%%Adult NO
%Anatomy

The noduli are small, bilaterally symmetric spherical neuropils located medially and ventrally to the FB. In the adult \textbf{Drosophila} brain, each hemilateral neuropil is divided in 3 subunits: nodulus 1, 2 and 3 (NO1, NO2, NO3), with NO1 having the highest synaptic density of the three. There are notable variations across insect species, with the number of noduli ranging from two to four per brain hemisphere.
While the stacked noduli subunits have been referred to as horizontal layers, no vertical subdivisions have been reported for these structures. Therefore there isn't any columnar organisation known.


The NO neurons present a unique morphology featuring compact, clutchy axons, which set them apart from other CX neurons~\citep{wolff2018neuroarchitecture}~\citep{hulse2021connectome} and greatly ease their identification even in the absence of the typical conspicuous anatomical neuropil region present in adult insects. In the adult fruit fly, these neurons primarily originate in the DM1, DM2 and DM3 lineages~\citep{andrade2019developmentally}.
%The primary neurites from G9–G6 cross the midline to arborize in the contralateral noduli.


At the larval stage of this animal, we found a set of neurons with highly compact, clutchy axons situated in the posterior ventral area of the brain, coming from lineages DM1 and DM3, as well as a few other larval lineages, and postulate this as the putative Noduli of the Drosophila larva. 

%Connectivity
In the adult \textit{Drosophila} brain, the NO is  interconnected with the EB and the FB, to which they relay information from tangential input neurons via several PB columnar cells such as PEN-neurons(PB-EB-NO; from the Head Direction System) and PFN-neurons(PB-FB-NO)~\citep{wolff2015neuroarchitecture, hulse2021connectome}. The primary NO inputs outside of the CX are from the LAL, these are known as LNO neurons and are suggested to be inhibitory~\citep{wolff2018neuroarchitecture,hulse2021connectome}. LNOs send inputs to and receive feedback from columnar neurons. %TODO check which comumnar neurons. 
FB tangential neurons make weak reciprocal connections to LNOs and columnar neurons in the NO.
NO is synaptically interconnected with the other CX neuropils. All columnar neurons (PFNs and PENS) that synapse onto NO (are NO.b) are recurrently connected to the same LNO neurons they receive input from. 

%TODO: figures with MB and other structures connecting to the CX. especially CNs connecting to every neuropil of the CX.  

In the putative larval NO, we find that the neurons projecting onto this neuropil receive input from LAL, (LAL.d MB2ON-75)

%Larval NO - 
%recurrent connections exist but they are axo-axonic - 
%most outputs are to MB2ONs, CNs, 10 pars of CNs, mostly MB related neurons. 
%inputs are mostly multicompartment MBONs. 
%TODO: no.d s should be the analogus to LNOs. 
%TODO: find all LAL NEURONS, pk_lal is the volume. LALbs and LALds neuron search : annotation - larval central complex, check everything that is LAL something . look at those that have fullly encolsed synapses -  the lal columns have boh dendrites and axons fully contained in the volumes 
%VMCc add it to the LAL. that is the LAL 


%The NO receives inputs from LNO neuron types that innervate accessory structures: LAL, GA, and CRE. 
% TODO check these set of synaptic connections
%The majority of NO outputs (of CX columnar neurons) are to other CX columnar neurons (usually of the same type), or to LNO neurons that then provide input to the CX columnar neurons.
%TODO: check if recurrence is axo axonic - 
%todo: check if LNO1,2,3 check if they are actually the same as NO.b 
%NO is synaptically interconnected with the other CX neuropils. All columnar neurons (PFNs and PENS) that synapse onto NO (are NO.b) are recurrently connected to the same LNO neurons they receive input from. 
%PEN and PFN send output and recieve inputs from NO
%LNOs mainly send outputs
%Important comment against NO being an output structure: The only CX columnar neurons that lend some credence to the notion of the NO being an output structure of the CX are the PEN_b neurons, which provide strong inputs to the ExR8 neurons ()


%%Sensory information
%Many of these columnar neurons likely also receive input related to the fly’s self-motion in paired structures known as the noduli. 
In the adult Drosophila, the NO receives optic flow-based self-motion information and wind direction information via the columnar neurons. %an important hub for self-motion information according to physiological and anatomical observations. 
% TODO In the larva, we find ... 16/21 receive inputs from MBONs, from PB (via 7 PB to noduli neurons, like in the adult), and from FB (some NO.b neurons are also FB.d neurons). 6/21 neurons are columanr neurons that aren't FB.d or PB.d or anything like that.

%%Larva
In Drosophila larva, we found a set of neurons with highly compact, clutchy axons situated in the posterior ventral area of the brain - similarly to the adult NO - coming from lineages DM1 and DM3, as well as a few other larval lineages. We observe that these neurons are highly interconnected with the PB and FB,  with strong inputs from 
PB and strong outputs to FB, and many of these neurons receive inputs in the LAL.
Their highly distinctive morphology, location as well as similarities in connectivity to the adult noduli, make these neurons an excellent candidate for the putative larval noduli.

%TO DO: can we see the kind of specific recurrent activity


\section{Neurotransmitter Identity of Central Complex Neurons}
\subsection*{DANs}
\subsection*{GABA and Acetylcholine}


\section{Functional Connectivity via Inhibition of CX neurons}
\subsection*{NBLAST genetic lines for CX neurons}
\subsection*{Behavioural Assays}





