% ************************** Thesis Abstract *****************************
% Use `abstract' as an option in the document class to print only the titlepage and the abstract.
\begin{abstract}

    \section*{Abstract}
     %The Central Complex (CX) is a conserved set of arthropod brain neuropils that integrates multisensory information and mediates spatial navigation and sleep. % different definition than in the introduction!
     In holometabolous insects such as the fruit fly \textit{Drosophila melanogaster}, the brain central complex~(CX) develops during metamorphosis and serves the adult stage. Whether a form of the CX exists in the brain of the evolutionarily novel larval stages is not known. Here, we analyzed the connectome of the larval brain and, on the basis of neuronal lineages, synaptic connectivity patterns, and anatomy, identified a putative larval CX, comprising 4 key neuropils: the protocerebral bridge~(PB), the ellipsoid body~(EB), the fan-shaped body~(FB) and the noduli~(NO). Consistent with our interpretation, we found in the larval brain synaptic connectivity patterns characteristic of the adult, including (i) visual input into the PB and EB; (ii) modulation of CX neuropil inputs by the mushroom body~(MB); (iii) reciprocal connectivity between CX neuropils and select MB compartments; and (iv) strong connectivity between CX neuropils.
     % FB and PB as well as FB and NO
     While some neuronal lineages contributing to the larval CX do not contribute to the adult CX, many others are conserved. The characterization of a larval CX brings structure to largely unexamined larval brain circuits, linking with a vast body of literature, and will inform the design of experiments to probe larval brain function.


\end{abstract}
