%!TEX root = ../thesis.tex
%*******************************************************************************
%*********************************** First Chapter *****************************
%*******************************************************************************

\chapter{Introduction}  



\section{Why is this Interesting?} 
\label{section1.1}
Framing the Question
\section{Mapping brains - The Power of Connectomics} 


\section{Multisensory integration} 

Interpreting sensory inputs requires dynamic transformations, the brain must integrate sensory cues into one coherent representation that maximizes sensitivity and reduces ambiguity, before turning it into action via
activation of specific muscles. This process is esential for maintaining goal-oriented behavioural programs over long timescales, which require the brain to ignore transient sensory distractions and compensate for fluctuation in the quality of information or its temporal availability.

Representations that persist in absence of sensory input rely on attractor dynamic generated by recurral neural circuits in the deep brain regions rather than at the sensory/motor periphery. Deep-brain circuits' inputs and outputs are usually difficult to identify and characterize, especially those involved in flexible navigation who's circuits have large populations of neurons.
The deep-layer connectivity is difficult to determine in large brain animals, such as mammals. Insects are better suited for this purpose, as they have small brains and identified neurons, providing the opportunity to obtain a detailed understanding of their circuits and how they generate behaviour. 

Insects maintain a specific pattern of action selection over many minutes and even hours during behaviors like foraging or migration, and maintain a prolonged state of inaction during quiet wakefulness or sleep (Hendricks et al., 2000; Shaw et al., 2000). Both types of behaviors are initiated and modulated based on environmental conditions (for example, humidity, heat, and the availability of food, nutritive state, hunger,hydration etc) and an insect’s internal needs (for example, sleep drive and nutritive state; Griffith, 2013). The context-dependent initiation and control of many such behaviors is thought to depend on a conserved insect brain region called the central complex (CX)


\section{The Central Complex}
The Central Complex(CX) is a morphologically conserved set of neuropils found across insects
%(bumblebee, the red flour beetle and the Monarch butterfly) 
that acts as navigational centre and sensory integration area for coordinated motor activity.
This region is best described %(connectivity wise) 
and understood %(function wise) 
in \textit{Drosophila} adult where its core functions include multisensory navigational decisions, path integration, allocentric orientation of the head relative to its body - convergence of head and body direction - and providing an internal sense of direction in the absence of stimuli. %also promotes sleep

At the larval stage, this animal exhibits  similar behaviours to those observed in the adult: it demonstrates chemotaxis during foraging and performs aversive phototaxis in response to blue light. In addition to individual stimulus response, \textit{Drosophila} larva is able to integrate competing stimuli into a coherent representation(Gepner et al.,2015) prior to decision making. The larval brain shares similar set of neuroblasts with the adult, and presents neuropils with direct correspondance to the adult such as the Antennal Lobe(AL), the Mushroom Body(MB), and the Lateral Accessory Lobe(LAL). 

The underlying connectivity of AL, MB, LAL is well understood at present(Winding et al., 2023), as well as the Lateral Horn(LH; a larvae specific neuropil). Nevertheless, these structures constitute only up to 25\% of the larval brain connectome. We postulate that amongst the remaining ~75\% of neurons, a multitude should be devoted to navigational decisions, and may constitute the putative larval Central Complex neuropils. These are unlikely to be reocgnizable morphologically at this stage of development, since the brain lobes aren't yet fused at the midline, and the larval brain presents a commisure. The basis of our search has to be, in turn, based on lineage membership, relative spatial location and circuit architecure specific to adult central complex neuropils. We use all three in an iterative process to progressively find putative CX neurons.


The adult \textit{\textit{Drosophila}} central complex has five neuropils: the Protocerebral Bridge (PB), the Fan-shaped Body (FB; or central body upper), the Ellipsoid body(EB; or central body lower), the Noduli (NO) ~\citep{hanesch1989neuronal} and, as of recently, the Assymetrical Body (AB) ~\citep{wolff2018neuroarchitecture}. The Lateral Accessory Lobe (LAL) reciprocally interconnects with these, making it an important accesory structure and reference point. 

The central complex has been associated with a set of functions - spatial navigation decisions, directed locomotion and sleep - some of which are shared by the larva. The neuroblasts that give rise to the neuronal lineages populating the adult CX also exist in the larval brain. A subset of embryonic-born neurons remain undifferentiated throughout larval stages and delineate the structures of the adult CX, acting as pioneer neurons during metamorphosis 
~\citep{andrade2019developmentally}; however, earlier-born, differentiated neurons of the same lineages contribute to structures in the larval brain. The question remains as to what structures. Furthermore, the larval brain presents readily recognizable neuropils of accepted homology with the adult brain, including the antennal lobe ~\citep{berck2016wiring}, mushroom body ~\citep{eichler2017complete}, and lateral accessory lobe ~\citep{hartenstein2015lineage}.
 In the adult brain, the central complex neuropils are primarily medial structures, suggesting that any putative larval counterpart will be necessarily split across the midline given the lack of fusion of the larval brain hemispheres. With all the above in mind, and considering the evolution of the larval stage in holometabolous insects ~\citep{truman1999origins}and the presence of a central complex-like structure in the larva of the holometabolous beetle Tribolium castaneum, we set out to identify the putative central complex neuropils of the fruit fly larva on the basis of: neurons contributing to the larval CX neuropils that share lineage of origin with the adult CX neurons;the synaptic connectivity present across the putative larval CX neuropils is at least a subset of that of the adult CX neuropils; the spatial position and overall morphology of the arbors of larval CX neuropils is similar to that of the adult CX neurons. 

\subsection*{Evolutionary origins of Central Complex}
Ancestrally, the central complex arises during embryogenesis in hemimetabolous insects - insects that undergo ‘incomplete’ metamorphosis, whose nymph stages transition slowly over multiple steps into the adult shape, with their brain developing gradually(e.g. beetle or ciccada).

In later derived holometabolous insects -  those that compress all nymph stages into a sessile stage, the pupa OR insects that undergo ‘complete metamorphosis’, and have distinct larval, pupal and adult stages designed to restrict developmental resources to those needed for growth - the central complex arises during  postembryonic stagese (e.g. flies,moths, bee).

In Drosophila,  the adult central complex starts forming during late larval and metamorphic stages, by the proliferation of pioneer undifferentiated neurons that remain quiescent until the pupal stage. This is in contrast to other holometabolous insects such as the tribolium (Farnworth et al., 2020) whose upper part begins forming during embryogenesis. 
    
What is special about the holometabolous insects such as drosophila is the extremely arrested brain development during larval stages the brain development during larval stages (Andrade et al, 2019), when they’re essentially free-living, feeding embryos(Truman et al., 1999).    
    
The rough connectivity patterns of the central complex neuropils are relatively well known, but what types of navigation require the central complex, whether most central complex neurons are multisensory, or how sensory information from different modalities is organized within the central complex is still not well understood. As stated by Currier et al. in 2020, in-depth studies combining single-cell manipulations of neural activity of gene expression, reconstructed electron microscopy circuits and behavioural assays should be used to better understand the wiring diagram responsible for multisensory interaction in the central complex.

\section{The insights connectomics can bring}

\section{The advantage of an insect brain}