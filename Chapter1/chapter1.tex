
\chapter{Introduction}  


\section{Mapping brains - The Power of Connectomics} 
\label{Mapping Brains}

Investigating the origins of behaviour and mapping it to the brain is a mission as old natural philosophy. Hippocrates was the first to identify the brain as the 'analyst of the outside world', the interpreter of consciousness and the center of intelligence and willpower \citep{breitenfeld2014hippocrates}, and to this day he is considered the forefather of neurology.

At its heart, neuroscience has always been about reverse engineering nervous systems. 
We attempt to understand the full functionality of the brain from its underlying anatomical features - molecular and cellular -  its activity and its circuits. Ultimately, the goal is to infer causality about behaviour using the best tools we have at our disposal \citep{mckinstry2013connectome}.

Detailed maps of synaptic connectivity (also known as wiring diagrams) are core to our understanding of the fundamental link between brain and behaviour and, crucially, how malfunctions of it can result in behavioral or neurological disorders. Whilst large-scale circuit reconstructions aren't yet possible with state-of-the-art imaging technology - the human brain has ~86 billion cells\cite{herculano2009human}, with tens of thousands of synapses each, and would take centuries to map -  advances in tiny brains mapping are pushing the brain sciences by unraveling causal and correlative relations between structure and function and paving the way to better understand fundamental principles of how neural systems operate. 

Connnectomics - the study of complete sets of connections in individual neural systems - is at large responsible for a lot of these discoveries. It allowed us to create roadmaps for brain of various animals such as the C. Elegans - the first landmark study of a connectome \citep{brenner1974genetics} -, the mouse visual cortex \citep{microns2025functional}, the \textit{Drosophila} adult \citep{scheffer2020connectome} as well as the \textit{Drosophila} larva\citep{ohyama2015multilevel, berck2016wiring,larderet2017opticlobe,eichler2017complete, eschbach2021circuits, eschbach2020recurrent}. All datasets were obtained via electron microscopy (EM, ssEM, TEM, FIB-SEM) and have become foundational references in neuroscience, ushering in the long-anticipated era of synaptic wiring diagrams (Table~\ref{table-connectomes}).

EM is, to this day, the highest-resolution technology for brain mapping, allowing us to visualize the smallest neurites (as small as 15~nm; \citep{meinertzhagen2016connectome}) and all the synaptic connections—vesicles and clefts (40~nm and 20~nm, respectively)—between them. The great advantage of tiny animal models such as \textit{Drosophila} is that their physical dimensions (~600~\textmu m or $0.15\,\mathrm{mm}^3$ for the adult fly brain; ~242~\textmu m or $0.001\,\mathrm{mm}^3$ for the larval nervous system) can be reconstructed with synaptic resolution within a manageable timeframe and cost. Their dimensions - no neuropil greatly exceeds 50~\textmu m in depth \citep{meinertzhagen2016connectome} — also facilitate the use of multimodal imaging approaches: confocal microscopy for molecular and genetic labeling, light-sheet microscopy for rapid volumetric imaging, and electron microscopy for ultrastructural detail—on the same specimen or across comparable samples.


        \begin{table}[t]
        \centering
        \small
        \setlength{\tabcolsep}{4pt}
        \renewcommand{\arraystretch}{1.1}
        \resizebox{\textwidth}{!}{%
        \begin{tabular}{l|cccccccc}
        \toprule
        Organism & Tissue / region & Neurons & Synapses & Voxel (nm) & Volume (mm$^3$) & Imaging method & Project \\
        \midrule
        \textit{C. elegans} & Entire nervous system & 302 & $\sim$7\,000 & $\sim$50 (2D TEM) & 0.01 & ssTEM & Original connectome \\
        \textit{D. melanogaster} (larva) & Entire CNS (brain + VNC) & $\sim$9\,700 & $\sim$5--10\,M & 3.8$\times$3.8$\times$50 & 0.001 & ssTEM & Larval CNS connectome \\
        \textit{D. melanogaster} (adult) & Entire brain (central + optic lobes) & $\sim$135\,000 & $\sim$50--60\,M & 4$\times$4$\times$40 & 0.15 & ssTEM & FAFB, Hemibrain \\
        \textit{M. musculus} & Visual cortex (VISp + HVAs) & $\sim$100\,M & $\sim$0.5\,B & 4$\times$4$\times$40 & 1 & SBF/MB-EM & MICrONS v3 \\
        \bottomrule
        \end{tabular}}
        \caption{Comparison of major connectomic datasets across species, from \textit{C. elegans} to mouse visual cortex. Abbreviations: ssTEM, serial-section TEM; SBF/MB-EM, serial block-face or multi-beam EM.}
        \label{table-connectomes}
        \end{table}


The female adult vinegar fly \textit{Drosophila Melanogaster} is one of the largest pieces of nervous tissue collected and imaged by EM (i.e., FAFB; \citep{zheng2018complete}). The dataset encompasses 40~teravoxels and 21~million serial section transmission EM images, covering a $995~\times~537~\times~283~\si{\micro\metre}$ volume at $4~\times~4~\times~40~\si{\nano\metre}$ resolution. Complementary to this, a complete EM reconstruction of the \textit{Drosophila} larva central nervous system (\citep{winding2023connectome}) spans the entire brain and ventral nerve cord within a roughly $242~\si{\micro\metre}$ volume, capturing approximately $9{,}700$~neurons at $3.8~\times~3.8~\times~50~\si{\nano\metre}$ resolution. I have contributed to mapping the larval fruit fly, specifically, in unraveling a new area responsible for navigational decisions. This area is known as the Central Complex in the adult, and, as I will go on to show, I postulate that an analogous, numerically smaller Central Complex exits at the larval stage of development. 

Together, these datasets bridge developmental stages of the fly nervous system, enabling both sparse and dense connectomic analyses that inform a growing range of hypotheses around their relevance to function and behaviour. 
Collectively, connectomic datasets provide a structural framework through which brain function can be interpreted. They transform anatomy into networks and reveal not only which neurons are connected but also the direction of information flow. This is the exact type of data that is translatable into connectivity matrices — quantitative maps of presynaptic and postsynaptic partnerships— that we can therefore use we can infer circuit logic and predict functional pathways to expose hidden motifs, highlight convergence or divergence across modalities, and delineates subnetworks anatomical areas that underlie specific behaviours. Using \textit{Drosophila} is powerful approach to functionally study these detailed wiring diagram, because it is one of the most experimentally accessible animal model.  


%This is exactly the kind of data one needs for collecting information about networks subnetworks submodalities connectivity matrices \citep{meinertzhagen2018use}. These connectomes allow neuroscientists to generate a map of the brain’s synaptic networks. The core scientific outcome of connectomics is the ability to compile comprehensive connectivity matrices by meticulously reconstructing neurons and their synaptic sites at the electron microscopic (EM) level. These matrices typically report the number of presynaptic and postsynaptic connections between partner neurons. By revealing unsuspected pathways and characterizing pathways as major (100 or more contacts) or minor (fewer than 10 contacts), connectomes provide the structural foundation necessary for interpreting functional data, fueling computational analyses, and enabling the prediction of function from anatomical structure. Ultimately, this capacity to map and quantify synaptic networks is essential for navigating the enormous complexity of the brain's circuitry and is destined to aid in understanding cognitive disorders having a connectomic basis. 

%While scientific advances haven't yet reached the human brain scale - since 1 mm³ takes about a couple of months to image, and the human brain has roughly 1.3 million mm³, that means it would take about 1.3 million years to have the entire brain mapped by EM; and that’s without reconstructing the connectome (86 billion neurons and about 100 trillion connections) 
%. It provides data that can be mined for both a parts-list and an adjacency matrix of connections between those parts. Other techniques, such as expansion microscopy (Wassie et al., 2019), super resolution microscopy (Igarashi et al., 2018) and molecular barcode sequencing (Kebschull and Zador, 2018) can provide sparse information on synaptic connectivity or map single neuron’s projections at scale. While they can provide orthogonal useful information, e.g. on gene expression, they lack the morphological and synaptic connectivity ‘completeness’ EM can provide. There is promise that X-ray tomography may provide another avenue to dense circuit reconstruction and allow researchers to use larger tissue volumes than can typically be easily handled with EM pipelines (Kuan et al., 2020).



\section{The advantage of the \textit{Drosophila} larva brain}
    tiny
\label{}



\section{Multisensory integration} 
    %why multisensory integration is a fundamental problem

    Interpreting sensory inputs requires dynamic transformations, the brain must integrate sensory cues into one coherent representation that maximizes sensitivity and reduces ambiguity, before turning it into action via
    activation of specific muscles. This process is essential for maintaining goal-oriented behavioural programs over long timescales, which require the brain to ignore transient sensory distractions and compensate for fluctuation in the quality of information or its temporal availability.

    Representations that persist in absence of sensory input rely on attractor dynamic generated by referral neural circuits in the deep brain regions rather than at the sensory/motor periphery. Deep-brain circuits' inputs and outputs are usually difficult to identify and characterize, especially those involved in flexible navigation whose circuits have large populations of neurons.
    The deep-layer connectivity is difficult to determine in large brain animals, such as mammals. Insects are better suited for this purpose, as they have small brains and identified neurons, providing the opportunity to obtain a detailed understanding of their circuits and how they generate behaviour. 


\section{Navigation as a mode to understand multisensory integration}

        Insects maintain a specific pattern of action selection over many minutes and even hours during behaviors like foraging or migration, and maintain a prolonged state of inaction during quiet wakefulness or sleep (Hendricks et al., 2000; Shaw et al., 2000). Both types of behaviors are initiated and modulated based on environmental conditions (for example, humidity, heat, and the availability of food, nutritive state, hunger,hydration etc) and an insect’s internal needs (for example, sleep drive and nutritive state; Griffith, 2013). The context-dependent initiation and control of many such behaviors is thought to depend on a conserved insect brain region called the central complex (CX) %Because multisensory integration is most clearly expressed in navigational behaviors, navigation offers a tractable mode to investigate these principles


\section{The Central Complex}

    \subsection{Central Comple across insects}
        The Central Complex(CX) is a morphologically conserved set of neuropils found across insects
        %(bumblebee, the red flour beetle and the Monarch butterfly) 
        that acts as navigational centre and sensory integration area for coordinated motor activity.

    \subsection{Evolutionary origins of Central Complex}
        Ancestrally, the central complex arises during embryogenesis in hemimetabolous insects - insects that undergo ‘incomplete’ metamorphosis, whose nymph stages transition slowly over multiple steps into the adult shape, with their brain developing gradually(e.g. beetle or ciccada).
        In later derived holometabolous insects -  those that compress all nymph stages into a sessile stage, the pupa OR insects that undergo ‘complete metamorphosis’, and have distinct larval, pupal and adult stages designed to restrict developmental resources to those needed for growth - the central complex arises during  postembryonic stagese (e.g. flies,moths, bee).

    \subsection{Central Complex in the Drosophila Adult}
        In Drosophila,  the adult central complex starts forming during late larval and metamorphic stages, by the proliferation of pioneer undifferentiated neurons that remain quiescent until the pupal stage. This is in contrast to other holometabolous insects such as the tribolium (Farnworth et al., 2020) whose upper part begins forming during embryogenesis. 
        What is special about the holometabolous insects such as drosophila is the extremely arrested brain development during larval stages the brain development during larval stages (Andrade et al, 2019), when they’re essentially free-living, feeding embryos(Truman et al., 1999).    
        The rough connectivity patterns of the central complex neuropils are relatively well known, but what types of navigation require the central complex, whether most central complex neurons are multisensory, or how sensory information from different modalities is organized within the central complex is still not well understood. As stated by Currier et al. in 2020, in-depth studies combining single-cell manipulations of neural activity of gene expression, reconstructed electron microscopy circuits and behavioural assays should be used to better understand the wiring diagram responsible for multisensory interaction in the central complex.

        This region is best described %(connectivity wise) 
        and understood %(function wise) 
        in \textit{Drosophila} adult where its core functions include multisensory navigational decisions, path integration, allocentric orientation of the head relative to its body - convergence of head and body direction - and providing an internal sense of direction in the absence of stimuli. %also promotes sleep

        The adult \textit{\textit{Drosophila}} central complex has five neuropils: the Protocerebral Bridge (PB), the Fan-shaped Body (FB; or central body upper), the Ellipsoid body(EB; or central body lower), the Noduli (NO) ~\citep{hanesch1989neuronal} and, as of recently, the Assymetrical Body (AB) ~\citep{wolff2018neuroarchitecture}. The Lateral Accessory Lobe (LAL) reciprocally interconnects with these, making it an important accesory structure and reference point. 
    
    \subsection{PB,EB,FB,NO}
    

\section{Larval Central Complex as a research opportunity}
    -developental neuronal diversity (the 3 types)

    At the larval stage, this animal exhibits  similar behaviours to those observed in the adult: it demonstrates chemotaxis during foraging and performs aversive phototaxis in response to blue light. In addition to individual stimulus response, \textit{Drosophila} larva is able to integrate competing stimuli into a coherent representation(Gepner et al.,2015) prior to decision making. The larval brain shares similar set of neuroblasts with the adult, and presents neuropils with direct correspondance to the adult such as the Antennal Lobe(AL), the Mushroom Body(MB), and the Lateral Accessory Lobe(LAL). 

    The underlying connectivity of AL, MB, LAL is well understood at present(Winding et al., 2023), as well as the Lateral Horn(LH; a larvae specific neuropil). Nevertheless, these structures constitute only up to 25\% of the larval brain connectome. We postulate that amongst the remaining ~75\% of neurons, a multitude should be devoted to navigational decisions, and may constitute the putative larval Central Complex neuropils. These are unlikely to be reocgnizable morphologically at this stage of development, since the brain lobes aren't yet fused at the midline, and the larval brain presents a commisure. The basis of our search has to be, in turn, based on lineage membership, relative spatial location and circuit architecure specific to adult central complex neuropils. We use all three in an iterative process to progressively find putative CX neurons.


    The central complex has been associated with a set of functions - spatial navigation decisions, directed locomotion and sleep - some of which are shared by the larva. The neuroblasts that give rise to the neuronal lineages populating the adult CX also exist in the larval brain. A subset of embryonic-born neurons remain undifferentiated throughout larval stages and delineate the structures of the adult CX, acting as pioneer neurons during metamorphosis 
    ~\citep{andrade2019developmentally}; however, earlier-born, differentiated neurons of the same lineages contribute to structures in the larval brain. The question remains as to what structures. Furthermore, the larval brain presents readily recognizable neuropils of accepted homology with the adult brain, including the antennal lobe ~\citep{berck2016wiring}, mushroom body ~\citep{eichler2017complete}, and lateral accessory lobe ~\citep{hartenstein2015lineage}.
    In the adult brain, the central complex neuropils are primarily medial structures, suggesting that any putative larval counterpart will be necessarily split across the midline given the lack of fusion of the larval brain hemispheres. With all the above in mind, and considering the evolution of the larval stage in holometabolous insects ~\citep{truman1999origins}and the presence of a central complex-like structure in the larva of the holometabolous beetle Tribolium castaneum, we set out to identify the putative central complex neuropils of the fruit fly larva on the basis of: neurons contributing to the larval CX neuropils that share lineage of origin with the adult CX neurons;the synaptic connectivity present across the putative larval CX neuropils is at least a subset of that of the adult CX neuropils; the spatial position and overall morphology of the arbors of larval CX neuropils is similar to that of the adult CX neurons. 


\section{What this thesis does}

This thesis aims to identify the putative central complex neuropils of the Drosophila larva, characterize their connectivity, neurotransmitter identity, and functional role in behavior, and place these findings in the context of central complex evolution and developmental diversity. To achieve this, I combined connectomic reconstructions, genetic tools, functional imaging, and behavioral assays. The thesis is structured as follows: Chapter 2 describes the methods… Chapter 3 presents results… Chapter 4 discusses…”
