%!TEX root = ../thesis.tex
%*******************************************************************************
%****************************** Third Chapter **********************************
%*******************************************************************************
\chapter{Discussion}
	Is there a Navigational Centre (Central Complex) in the Larval Drosophila?

	The similarities between the larval Central Complex and the adult Central Complex
        PB,EB,FB,NO

        Relation to accessory structures such as the LAL 
        Sensory Biases: olfaction vs photoreception
        Central Complex relation to the Mushroom Body (Learning and Memory Centre of the fly)
iii.	Functions of the Larval central complex 
1.	Network connectivity predictions
2.	Does biological data confirm our intuition?
3.	Conclusions about the neural functionality
iv.	What we now know
v.	Expectations for the future
vi.	Can this teach us universal components of multisensory integration 


\section{Multisensory integration}
 Gepner et al., 2015 demonstrated that Larvae  have to integrate visual and olfactory gradients, with convergence of these sensory systems before decision to act on them. 
 Exposed larvae to visual and olfactory input - blue light and optogenetic activation of appetitive ORNs
No competition between types of taxis. 
multisensory integration happens immediately before the decision to navigate. Only one coherent representation
Why is it that the information isn’t processed at the time of decision making  - i guess because command like neurons 


\section{}
%%%%PAPER CONTENT%%%


Our inquire into the nature of a fraction of neurons in the remaining \textit{terra incognita} of the larval brain of the fruit fly yielded a series of neurons, neuropils and circuits with undeniable developmental, connectivity, and anatomical similarities to the set of core neuropils of the central complex of the adult fly brain.
The key anatomical difference of split brain hemispheres and the very early arrest of neuroblast proliferation in the larvae of Diptera impose strong constraints as to the extent of the anatomical and cellular similarities between the larval and the adult brain.

Notwithstanding, most larval brain neurons survive metamorphosis, help articulate homologous brain neuropils in the adult \citep{prieto2012embryonic}, and participate of neural circuits in the adult brain. For example, the command neurons for backward locomotion \citep{carreira2018mdn}, and MB MBINs and MBONs \citep{truman2023metamorphosis}.

% DISCUSS the Truman papers on larva being an evolutionary novelty whose brain recruited neurons onto temporary identities and which revert to the ancestral identity in the pupa and adult.
Intriguingly, while we find that larval brain neuronal lineages contribute the same neuronal cell types to the same CX neuropils as their corresponding adult brain instances do, we also find a number of neurons contributing to the larval CX that originate in non-CX lineages. This pattern has been reported before for the MB \citep{truman2023metamorphosis}, with some larval MB neurons later on either undergoing apoptosis or, astonishingly, contributing to CX neuropils after metamorphosis. With the MB being an undeniably homologous structure to that of the adult and the larval Kenyon cells making up the bulk of the adult fly brain's gamma Kenyon cells, the recruitment of non-CX lineages into contributing CX neurons in the larva is perhaps part of a common theme in an animal with a limited neuron budget. The broader context being that the larva of the holometabola is derived \citep{truman1999origins, truman2019evolution}, an evolutionary novelty that enabled this group of animals to exploit novel ecological niches that, to boot, avoids competition with the adults.
In such predicament, it is perhaps not surprising that other larval neuropils similarly to the MB temporarily adopt neurons as their own that will eventually either perish by apoptosis or return to an ancestral cell fate away from the CX.


\subsection{Navigation, place learning, and memory}
% DISCUSSION material: working memory, or very short-term memory, related to temporal comparisons across head casting
% TODO a bit of context and introdution to head casting, sampling across left-right but also across time to determine direction of gradients
% Include discussion on eDANs (non-MB DANs) and on DNs to drive behavior – TODO pending

The central complex has been associated with place learning (for review see \citep{PfeifferHomberg2014}), with the EB in particular being necessary for visual place learning in flies \citep{ofstad2011visual}. With fly larvae being capable of local search behavior \citep{kromp2024localsearch}, the structure of the larval central complex and its close interactions with the center for associative memory, the mushroom body, must be examined.

We reported that the 'c' compartment of the larval MB is intimately associated with the FB. Intriguingly, MBON-c1 lacks a known role in associative memory to date, with all tests having been performed for olfactory associative memory \citep{eschbach2020recurrent}. Likewise, we discovered a tight loop between the NO and the MB vertical lobe, by means of neurons that also lack a known function in learning: the multi-compartment MBON-p1, whose dendrites span across the MB vertical lobe, and MBIN-l1, a MB input neuron of unknown neurotransmitter signature and function. Both of these MB compartments ought to be examined experimentally for potential roles in navigation or place learning.

With the NO having a role in the adult fly brain controlling the time course of walking activity \citep{strauss1993higher} and in handedness in locomotion \citep{buchanan2015neuronal}, and the larval MB vertical lobe mediating aversive memories \citep{eschbach2020recurrent}, our report of a tight loop between the output of the MB vertical lobe and modulatory input onto it via the NO may provide the necessary corollary discharge to adjust MB operations to imminent locomotor activity, potentially laterally biased since both MBON-p1 and MBIN-l1 are both entirely ipsilateral-only neurons.

Furthermore, we found that the 'g' compartment of the MB, whose MBON-g1 and MBON-g2 promote avoidance \citep{eschbach2021circuits} and its DAN-g1 is implicated in aversive learning \citep{eschbach2020recurrent}, is intimately and recurrently associated with the Wedge and Ring neurons of the EB. Two further MBONs, MBON-a2 and MBON-b3, likewise synapse onto EB neurons. Among these, the function of MBON-b3 remains unknown. Intriguingly, both MBON-a2 and MBON-b3 are rare among the MBONs in having an ipsilateral-only dendrite but a bilateral axon, when the arborization pattern of MBONs is most often the opposite: bilateral dendrites and an ipsilateral axon. The ability of these MBONs to differentially discern sensory inputs between the left and right sides of the body is suggestive of a role in lateralized behavioral responses, such as in taxis.

In loss-of-function experiments, the suppression of DAN-d1 with TNT removes excitatory drive from MBON-d1, which is GABAergic \citep{eschbach2021circuits}. From the connectome we interpret that a neuron postsynaptic to MBON-d1, MB2ON-63, which is a PB horizontal fiber distinct from DALv1, will then receive less GABAergic input. And therefore, MB2ON-63 receives unopposed excitatory input from visual PNs (PVL09) and other excitatory visual neurons (ChalOLP), and from MBON-c1, all of which are cholinergic \citep{larderet2017opticlobe, eschbach2021circuits}. We speculate that, if in wild-type conditions the inhibition from the MB via MBON-d1 was potentially subtracting an expected intensity of light, to perform a temporal comparison, without it the animal would no longer be able to assert where the light has increased or decreased, leading to the need for continuing to sample by head casting or turning.


\subsection{DANs in the FB and PB}

In the adult CX, dFB is a DAN that releases dopamine at the dorsal FB and is capable of triggering sleep~\citep{pimentel2016sleep}. Intriguingly, we found a DAN (eDAN-2) among the FB intrinsic neurons, and two additional DANs (eDAN-4l and eDAN-4m) that synapse onto a number of FB.d, FB.b and NO.d neurons. Identifying genetic driver lines for these neurons will enable testing experimentally their potential role in sleep regulation in larvae.

The role of eDAN-1, projecting to the larval PB, could be related to overall responsiveness and alertness of the animal, since there is a report that DANs projecting to the adult PB mediate increases in aggressiveness~\citep{alekseyenko2013DAN_PB}, and other DANs also projecting to the adult PB decrease sleep~\citep{tomita2021DAN_PB}.

What any of the identified non-MB DANs do awaits the identification of specific genetic driver lines and appropriate experimental setups to study their function.

\subsection{Conclusion}

In conclusion, our interpretation of some of the until now underexamined larval brain circuits as a numerically reduced version of the adult fly central complex is coherent with the evolution of the larval stage in the Holometabola and with the known cell types and overall synaptic connectivity of the corresponding neuropils in adult fly brain, bringing a whole field of study into a life stage of reduced dimensions and numbers of neurons and synapses. Now, with our identification of tight circuit loops between understudied MB compartments and the larval CX, an opportunity opens to examine the neural circuit basis of spatial navigation and place learning in this experimentally tractable animal.
