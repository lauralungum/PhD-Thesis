%!TEX root = ../thesis.tex
%*******************************************************************************
%****************************** Third Chapter **********************************
%*******************************************************************************
\chapter{Discussion}

	%The similarities between the larval Central Complex and the adult Central Complex
        %Relation to accessory structures such as the LAL 
        %Sensory Biases: olfaction vs photoreception
        %Central Complex relation to the Mushroom Body (Learning and Memory Centre of the fly)
%iii.	Functions of the Larval central complex 
 %       1.	Network connectivity predictions
%        2.	Does biological data confirm our intuition?
%        3.	Conclusions about the neural functionality
%        iv.	What we now know
%        v.	Expectations for the future
%        vi.	Can this teach us universal components of multisensory integration 



\section{Is there a Navigational Centre (Central Complex) in the Larval \textit{Drosophila}?}
Our inquire into the nature of a fraction of neurons in the remaining \textit{terra incognita} of the larval brain of the fruit fly yielded a series of neurons, neuropils and circuits with undeniable developmental, connectivity, and anatomical similarities to the set of core neuropils of the central complex of the adult fly brain.
The key anatomical difference of split brain hemispheres and the very early arrest of neuroblast proliferation in the larvae of Diptera impose strong constraints as to the extent of the anatomical and cellular similarities between the larval and the adult brain.

Notwithstanding, most larval brain neurons survive metamorphosis, help articulate homologous brain neuropils in the adult \citep{prieto2012embryonic}, and participate to neural circuits in the adult brain. For example, the command neurons for backward locomotion \citep{carreira2018mdn}, and MB MBINs and MBONs \citep{truman2023metamorphosis}.

Intriguingly, while we find that larval brain neuronal lineages contribute the same neuronal cell types to the same CX neuropils as their corresponding adult brain instances do (9 of 15 lineages), we also find a number of neurons contributing to the larval CX that originate in non-CX lineages. This pattern has been reported before for the MB \citep{truman2023metamorphosis}, with some larval MB neurons later on either undergoing apoptosis or, astonishingly, contributing to CX neuropils after metamorphosis. With the MB being an undeniably homologous structure to that of the adult and the larval Kenyon cells making up the bulk of the adult fly brain's gamma Kenyon cells, the recruitment of non-CX lineages into contributing CX neurons in the larva is perhaps part of a common theme in an animal with a limited neuron budget. The broader context being that the larva of the holometabola is derived \citep{truman1999origins, truman2019evolution}, an evolutionary novelty that enabled this group of animals to exploit novel ecological niches that, to boot, avoids competition with the adults.
In such predicament, it is perhaps not surprising that other larval neuropils similarly to the MB temporarily adopt neurons as their own that will eventually either perish by apoptosis or return to an ancestral cell fate away from the CX.


\subsection{Larval CX presents  similarities to the Adult CX}

        \subsubsection{PB}
        The Protocerebral Bridge (PB) serves as a critical hub for processing navigational signals in both larval and adult \textit{Drosophila}, with remarkable conservation of lineage origins and circuit architecture. In the adult, PEN (PB.d EB.b NO.b) columnar neurons are crucial components of the ring attractor network and they synapse bidirectionally onto PB horizontal neurons (EPG, EB.d PB.b GA.b) to adjust head direction based on angular velocity.  The larval PB similarly features columnar neurons (a PB.d NO.b and a PB.d EB.d LAL.b) that derive from homologous developmental lineages (DPMpm1/2, DPMm1, CM4), and that receive input from horizontal PB neurons (HF-PB) that make via axo-dendritic synapses. 

        Both developmental stages show tight integration between the PB and associative memory circuits: the adult PB receives modulatory dopaminergic (LPsP) and octopaminergic (P1-9) input, while larval PB horizontal fibres integrate multisensory information(visual and olfactory) that has been processed through MB-associated Convergence Neurons (CNs), in addition to receiving dopaminergic input via eDAN-1 (which is itself a PB.b) eDAN-2, eDAN-6 and eDAN-7. This architectural conservation suggests the PB's fundamental role in coordinating navigational computations could have been maintained despite the dramatic reorganization of the nervous system during metamorphosis. 


        \subsubsection{EB}
        The core connectivity between Ring and Wedge neurons within the larval EB is highly reminiscent of the adult circuit organization:

        Wedge Network: Larval EB wedge neurons (W1–W8) form a densely recurrent network, resembling adult wedge EPG neurons. The predominant connection motif is axo-dendritic synapses, although weaker axo-axonic contacts are also present.
        Wedge-to-Ring Connectivity: All eight larval wedge neuron pairs synapse directly onto several identified ring neuron (RN1/2a,b and RN3), creating an all-to-all Wedge to Ring connectivity motif that closely mirrors the characteristic pattern observed in the adult EB. However, these connections are exclusively axo-axonic.
         Ring Interconnectivity: Larval ring neurons (RN1/2, RN3) are reciprocally connected axo-axonically, a feature consistent with the dense interconnectivity seen among nearly all adult ring neuron types
         Another interesting similarly with adult is the dopaminergic input into  wedge neurons. The adult EPG wedge neurons are modulated by dopamine \citep{frighetto2022dopamine} , and we see there is a direct synapse from e-DAN3 onto the larval W-1 neuron. 

         Lineages: RNs from BAMv1/2(LALv1) and from EBa1 (DALv23), lineage CM1/3 (DM5/DM6).  wedge neurons from DALv23 
        
        \subsubsection{FB}
        FB tangential neurons hs-FB receive direct input from eDAN-2, eDAN-4l and eDAN-4m. Similarly in the adult 2 DANs (from PPL cluster)  provide input to FB tangential neurons to regulate sleep-wake cycle \citep{liu2012two}. 

        A key, evolutionarily conserved feature is the association with the Mushroom Body (MB), which is the center for associative memory: in both the adult FB (via tangential neurons targeting middle layers) and the larval FB (via tangential neurons hs-FBs and intrinsic neuron), neurons receive numerous synaptic inputs both indirectly from MB related (MB2ONs, MB2INs and CNs) and directly from MBON-c1; a pattern characteristic of the adult CX . Internally, both FBs are characterized by dense and highly recurrent networks composed of columnar cells (PB-FB-*, hDelta, and vDelta types in the adult) and tangential cells, although the larval structure lacks the clear layering and high neuronal diversity seen in the adult. Furthermore, both stages function as essential hubs for multimodal sensory integration, consistently processing visual inputs (via VPNs and VLNs in the larva) and olfactory cues (often routed via the Lateral Horn and MB outputs through Convergence Neurons), supporting the FB's role in flexible, state-dependent behaviour.

        Intriguingly, adult Fan‑Shaped Body (FB) tangential neurons (FBt) are known to synapse onto FB columnar neurons—notably vDelta, hDelta and PFL types—and this same tangential to columnar motif is evident in the larval FB. All identified hs‑FB tangential neurons in the larva make synaptic contacts onto at least one FB.d columnar neuron type.

        The larval FB shares several lineages with the adult FB neurons. The DM1-DM4 (larval lineages DPMm1, DPMpm1, DPMpm2, and CM4, respectively)  lineages that contribute to the adult CX also generate a number of FB tangential neurons (hs-FBs) and 5 pairs of FB columnar neurons in the larva. In addition the DALcm12 lineage (adult CREa1) contributes 1 pair of of columnar FBs (FB.d5)

        \subsubsection{NO}
        The most striking feature of the larval noduli is the morphology of its neurons' compact clutchy axons, which is a characteristic feature of the adult NO. Since the NO in the adult is by default split across the two hemispheres, it actually is plausible that such a structure could be conserved at an earlier stage of this animal.  In addition, overlapping contributing developmental lineages include DM1 (DPMm1), DM2 (DPMpm1), DM3 (DPMpm2) and SIPp1 (DPMl12), supporting a shared ontogenetic origin with the adult NO.



\subsection{Navigation, place learning, and memory}
% DISCUSSION material: working memory, or very short-term memory, related to temporal comparisons across head casting
% TODO a bit of context and introdution to head casting, sampling across left-right but also across time to determine direction of gradients
% Include discussion on eDANs (non-MB DANs) and on DNs to drive behavior – TODO pending

The central complex has been associated with place learning (for review see \citep{PfeifferHomberg2014}), with the EB in particular being necessary for visual place learning in flies \citep{ofstad2011visual}. With fly larvae being capable of local search behaviour \citep{kromp2024localsearch}, the structure of the larval central complex and its close interactions with the center for associative memory, the mushroom body, must be examined.

We reported that the 'c' compartment of the larval MB is intimately associated with the FB. Intriguingly, MBON-c1 lacks a known role in associative memory to date, with all tests having been performed for olfactory associative memory \citep{eschbach2020recurrent}. Likewise, we discovered a tight loop between the NO and the MB vertical lobe, by means of neurons that also lack a known function in learning: the multi-compartment MBON-p1, whose dendrites span across the MB vertical lobe, and MBIN-l1, a MB input neuron of unknown neurotransmitter signature and function. Both of these MB compartments ought to be examined experimentally for potential roles in navigation or place learning.

With the NO having a role in the adult fly brain controlling the time course of walking activity \citep{strauss1993higher} and in handedness in locomotion \citep{buchanan2015neuronal}, and the larval MB vertical lobe mediating aversive memories \citep{eschbach2020recurrent}, our report of a tight loop between the output of the MB vertical lobe and modulatory input onto it via the NO may provide the necessary corollary discharge to adjust MB operations to imminent locomotor activity, potentially laterally biased since both MBON-p1 and MBIN-l1 are entirely ipsilateral-only neurons.

Furthermore, we found that the 'g' compartment of the MB, whose MBON-g1 and MBON-g2 promote avoidance \citep{eschbach2021circuits} and its DAN-g1 is implicated in aversive learning \citep{eschbach2020recurrent}, is intimately and recurrently associated with the Wedge and Ring neurons of the EB. Two further MBONs, MBON-a2 and MBON-b3, likewise synapse onto EB neurons. Among these, the function of MBON-b3 remains unknown. Intriguingly, both MBON-a2 and MBON-b3 are rare among the MBONs in having an ipsilateral-only dendrite but a bilateral axon, when the arborization pattern of MBONs is most often the opposite: bilateral dendrites and an ipsilateral axon. The ability of these MBONs to differentially discern sensory inputs between the left and right sides of the body is suggestive of a role in lateralized behavioral responses, such as in taxis.

In loss-of-function experiments, the suppression of DAN-d1 with TNT removes excitatory drive from MBON-d1, which is GABAergic \citep{eschbach2021circuits}. From the connectome we interpret that a neuron postsynaptic to MBON-d1, MB2ON-63, which is a PB horizontal fiber distinct from DALv1, will then receive less GABAergic input. And therefore, MB2ON-63 receives unopposed excitatory input from visual PNs (PVL09) and other excitatory visual neurons (ChalOLP), and from MBON-c1, all of which are cholinergic \citep{larderet2017opticlobe, eschbach2021circuits}. We speculate that, if in wild-type conditions the inhibition from the MB via MBON-d1 was potentially subtracting an expected intensity of light, to perform a temporal comparison, without it the animal would no longer be able to assert where the light has increased or decreased, leading to the need for continuing to sample by head casting or turning.


\subsection{DANs in the FB and PB}

In the adult CX, dFB is a DAN that releases dopamine at the dorsal FB and is capable of triggering sleep~\citep{pimentel2016sleep}. Intriguingly, we found a DAN (eDAN-2) among the FB intrinsic neurons, and two additional DANs (eDAN-4l and eDAN-4m) that synapse onto a number of FB.d, FB.b and NO.d neurons. Identifying genetic driver lines for these neurons will enable testing experimentally their potential role in sleep regulation in larvae.

The role of eDAN-1, projecting to the larval PB, could be related to overall responsiveness and alertness of the animal, since there is a report that DANs projecting to the adult PB mediate increases in aggressiveness~\citep{alekseyenko2013DAN_PB}, and other DANs also projecting to the adult PB decrease sleep~\citep{tomita2021DAN_PB}.

What any of the identified non-MB DANs do awaits the identification of specific genetic driver lines and appropriate experimental setups to study their function.

%DANS in the adult 
% https://www.researchgate.net/figure/Wiring-Diagram-of-PB-Network-A-Generalized-anatomy-of-a-PB-glomerulus-using-PB-R1-as_fig20_236938211
%     % also - a dopaminergic pathway formed by large field CIVP neurons that relay IDFP signals to the entire PB

\subsection{Conclusion}

In conclusion, our interpretation of some of the until now underexamined larval brain circuits as a numerically reduced version of the adult fly central complex is coherent with the evolution of the larval stage in the Holometabola and with the known cell types and overall synaptic connectivity of the corresponding neuropils in adult fly brain, bringing a whole field of study into a life stage of reduced dimensions and numbers of neurons and synapses. Now, with our identification of tight circuit loops between understudied MB compartments and the larval CX, an opportunity opens to examine the neural circuit basis of spatial navigation and place learning in this experimentally tractable animal.









\section{Multisensory integration}
 Gepner et al., 2015 demonstrated that Larvae  have to integrate visual and olfactory gradients, with convergence of these sensory systems before decision to act on them. 
 Exposed larvae to visual and olfactory input - blue light and optogenetic activation of appetitive ORNs
No competition between types of taxis. 
multisensory integration happens immediately before the decision to navigate. Only one coherent representation
Why is it that the information isn’t processed at the time of decision making  - i guess because command like neurons 


 PB Modulation: In the larva, multi-sensory convergence onto PB horizontal fibers (DALv1 neurons) is directly modulated by the MB via Convergence Neurons (CNs) (e.g., CN-53, CN-54), which integrate both LH (innate) and MBON (associative memory) inputs